\documentclass{article}
\usepackage[utf8]{inputenc}
\usepackage{geometry}
\usepackage{enumitem}
\usepackage{graphicx}
\usepackage{listings}

\graphicspath{ {./images/} }

\geometry{a4paper, total={170mm,257mm}, left=20mm, top=20mm}

\begin{document}

\begin{center}
\textbf{\Large Verifica di \\ Tecnologie e progettazione di sistemi informatici e di telecomunicazioni}
\end{center}

\begin{flushleft}
Nome Cognome: \underline{\hspace{5cm}} \quad Classe: \underline{\hspace{3cm}} \quad Data: \underline{\hspace{3cm}}
\end{flushleft}

\section*{Esercizio 1}

\textbf{Scrivi il codice JS essenziale per implementare, in ordine, le seguenti:}

\begin{itemize}
    \item una funzione \texttt{contaVocali} che restituisca il numero totale di vocali minuscole o maiuscole di una stringa ricevuta come parametro;
    \item una funzione \texttt{maxVocali} che, ricevute due stringhe come parametri, restituisca \texttt{true} se la prima contiene più vocali della seconda, \texttt{false} altrimenti;
    \item una funzione \texttt{arrMaxVocali} che, ricevuto un array di stringhe come parametro, restituisca la stringa con più vocali di tutte le altre.
\end{itemize}

\textbf{N.B. per semplicità, si ignorino le vocali accentate}

\begin{flushright}\textbf{Punteggio:} \underline{\hspace{1cm}}/6\end{flushright}

\section*{Esercizio 2}

\textbf{Osserva il codice riportato sotto e rispondi alle domande}

\begin{lstlisting}
    function makeSmth(myWords, yourWord) {
        const c = 0;
        for (let d=0; d < myWords.length; d++) {
            if (myWords[d].length < yourWord.length) {
                c++;
            }
        }
        return c;
    }
\end{lstlisting}

\begin{itemize}
    \item Qual è l'errore che va obbligatoriamente corretto perché la funzione possa svolgere ciò per cui è stata scritta?
    \item Una volta corretta, qual è il comportamento implementato da questa funzione? Spiega
    \item Parlando di DOM, se \texttt{yourWord} fosse il riferimento a un campo di input testuale, cosa dovresti modificare nella funzione per ottenere lo stesso comportamento?
\end{itemize}

\begin{flushright}\textbf{Punteggio:} \underline{\hspace{1cm}}/6\end{flushright}

\section*{Esercizio 3}

\textbf{Parti dal codice HTML qui riportato e soddisfa le richieste presenti sotto:}

\begin{lstlisting}
<!DOCTYPE html>
<html>
<head>
    <title>Test</title>
</head>
<body>
    <h1>Test</h1>
    <p id="paragrafo1">Un paragrafo.</p>
    <p class="testo">Un altro paragrafo.</p>
    <div>
        <ul>
            <li>Elemento 1</li>
            <li class="elemento">Elemento 2</li>
            <li id="elemento3">Elemento 3</li>
        </ul>
        <span id="output"></span>
    </div>
    <button onclick="testFunzione()">Clicca qui</button>
    <script>
        function testFunzione() {
            // La risposta viene inserita qui! 
            // Nessuna modifica alle altre righe necessaria
        }
    </script>
</body>
</html>
\end{lstlisting}

\begin{itemize}
    \item scrivi il codice JS che inseriresti nel corpo della funzione di test per stampare, nell'elemento con id "output", il numero di $<li>$ con classe "elemento" della pagina;
    \item scrivi il codice JS che inseriresti nel corpo della funzione di test per modificare, a piacere, almeno tre proprietà CSS dell'elemento con id "paragrafo1".
\end{itemize}

\begin{flushright}\textbf{Punteggio:} \underline{\hspace{1cm}}/6\end{flushright}



\section*{Esercizio 4}

\textbf{Sempre in riferimento al codice HTML dell'esercizio 3, aggiungendo un secondo tag script subito dopo quello già presente, soddisfa la seguente descrizione utilizzando solo JS:}

\bigbreak

si vuole un bottone contenente la scritta "Wow!" che, quando premuto, faccia sì che tutti i $<li>$ della pagina vedano il loro testo interno trasformarsi in un collegamento ipertestuale a https://www.w3schools.com.


\begin{flushright}\textbf{Punteggio:} \underline{\hspace{1cm}}/6\end{flushright}

\section*{Esercizio 5}

\textbf{Immagina una pagina HTML contenente una lista di questo tipo:}

\begin{lstlisting}
[...]
<ul id="nomi">
    <li>Adalberto</li>
    <li>Bartolomeo</li>
    <li>Cipriano</li>
    <li>Domenico</li>
    <li>Evaristo</li>
    <li>Ferdinando</li>
    <li>Gerolamo</li>
    <li>Ludovico</li>
    <li>Olimpio</li>
    <li>Quintino</li>
    <!-- potrebbe continuare ... -->
</ul>
[...]
\end{lstlisting}

\textbf{Implementa uno script in grado di riordinare e riscrivere il contenuto della lista, qualsiasi esso sia e non solo per i nomi sopra, sulla base del numero di vocali contenute in ciascun elemento (es. dal nome con più vocali a quello con meno vocali). Puoi ovviamente riutilizzare le funzioni scritte nell'esercizio 1.}

\begin{flushright}\textbf{Punteggio:} \underline{\hspace{1cm}}/6\end{flushright}

% Aggiungi ulteriori sezioni o esercizi come necessario

\end{document}