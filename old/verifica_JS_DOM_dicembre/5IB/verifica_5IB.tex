\documentclass{article}
\usepackage[utf8]{inputenc}
\usepackage{geometry}
\usepackage{enumitem}
\usepackage{graphicx}
\usepackage{listings}

\graphicspath{ {./images/} }

\geometry{a4paper, total={170mm,257mm}, left=20mm, top=20mm}

\begin{document}

\begin{center}
\textbf{\Large Verifica di \\ Tecnologie e progettazione di sistemi informatici e di telecomunicazioni}
\end{center}

\begin{flushleft}
Nome Cognome: \underline{\hspace{5cm}} \quad Classe: \underline{\hspace{3cm}} \quad Data: \underline{\hspace{3cm}}
\end{flushleft}

\section*{Esercizio 1}

\textbf{Scrivi il codice JS essenziale per implementare, in ordine, le seguenti:}

\begin{itemize}
    \item una funzione \texttt{clean} che riceva un parametro stringa \texttt{s} e un parametro \texttt{x} e provveda a restituire la versione pulita di \texttt{s} (in cui non compaiono mai occorrenze di \texttt{x}, che si ipotizza essere un singolo carattere);
    \item una funzione \texttt{cleanArr} che, ricevuto un array di stringhe come parametro e un parametro \texttt{x}, restituisca un nuovo array contenente tutte le stringhe pulite da \texttt{x};
    \item una funzione \texttt{isCleanArr} che, ricevuto un array di stringhe come parametro e un parametro \texttt{x}, restituisca \texttt{true} se non ci sono elementi da pulire, \texttt{false} altrimenti.
\end{itemize}

\begin{flushright}\textbf{Punteggio:} \underline{\hspace{1cm}}/6\end{flushright}


\section*{Esercizio 2}

\textbf{Osserva il codice riportato sotto e rispondi alle domande}

\begin{lstlisting}
    function boh(numbers, d) {
        let r = 0;
        for (let i = d; i < numbers.length; i++) {
            let r += numbers[i];
        }
        return r;
    }
\end{lstlisting}

\begin{itemize}
    \item Qual è l'errore che va obbligatoriamente corretto perché la funzione possa svolgere ciò per cui è stata scritta?
    \item Una volta corretta, qual è il comportamento implementato da questa funzione? Spiega.
    \item Parlando di DOM, se \texttt{d} fosse il riferimento a un tag $<span>$ con id "val", cosa dovresti modificare nella funzione per ottenere lo stesso comportamento?
\end{itemize}

\begin{flushright}\textbf{Punteggio:} \underline{\hspace{1cm}}/6\end{flushright}

\section*{Esercizio 3}

\textbf{Parti dal codice HTML qui riportato e soddisfa le richieste presenti sotto:}

\begin{lstlisting}
<!DOCTYPE html>
<html>
<head>
    <title>E-commerce</title>
</head>
<body>
    <h1>Catalogo Prodotti</h1>
    <p id="descrizione">Sezione dei prodotti disponibili.</p>
    <p class="highlight">Offerte speciali!</p>
    <div>
        <ol id="elenco">
            <li>Prodotto A</li>
            <li class="offerta">Prodotto B</li>
            <li id="prodotto3">Prodotto C</li>
        </ol>
        <span id="risultato"></span>
    </div>
    <button onclick="faiQualcosa()">Aggiorna</button>
    <script>
        function faiQualcosa() {
            // La risposta viene inserita qui! 
            // Nessuna modifica alle altre righe necessaria
        }
    </script>
</body>
</html>
\end{lstlisting}

\begin{itemize}
    \item Scrivi il codice JS che inseriresti nel corpo della funzione per visualizzare, nell'elemento con id "risultato", il numero di paragrafi presenti nella pagina.
    \item Scrivi il codice JS che inseriresti nel corpo della funzione per modificare, a piacere, almeno tre proprietà CSS dei figli dell'elemento con id "elenco".
\end{itemize}

\begin{flushright}\textbf{Punteggio:} \underline{\hspace{1cm}}/6\end{flushright}

\section*{Esercizio 4}

\textbf{Sempre in riferimento al codice HTML dell'esercizio 3, aggiungendo un secondo tag script subito dopo quello già presente, soddisfa la seguente descrizione utilizzando solo JS:}

\bigbreak

si vuole aggiungere un div di dimensioni 100x100 pixel contenente la scritta "magico" che faccia sparire i figli della lista con id "elenco" quando l'utente passa il mouse sopra alla sua area (una volta fuori dall'area del div, gli elementi della lista tornano visibili).

\begin{flushright}\textbf{Punteggio:} \underline{\hspace{1cm}}/6\end{flushright}

\section*{Esercizio 5}

\textbf{Immagina una pagina HTML contenente una lista di questo tipo:}

\begin{lstlisting}
[...]
<ul id="incipit">
    <li>Nel mezzo del cammin di nostra vita</li>
    <li>Quel ramo del lago di Como</li>
    <li>Molti anni dopo, di fronte al plotone di esecuzione</li>
    <li>Cantami, o Diva, del pelide Achille</li>
    <!-- potrebbe continuare ... -->
</ul>
[...]
\end{lstlisting}

\textbf{Implementa uno script in grado di sostituire ogni incipit della lista con il corrispondente numero di spazi contenuti. Potresti anche aiutarti con le funzioni scritte nell'esercizio 1.}

\begin{flushright}\textbf{Punteggio:} \underline{\hspace{1cm}}/6\end{flushright}



% Aggiungi ulteriori sezioni o esercizi come necessario

\end{document}